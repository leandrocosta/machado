Frequent pattern mining is one of the most exploited subjects in data mining, assuming a key role in numerous tasks that have the goal of finding patterns of interest in a given data set. However, most of the solutions proposed in this line of research still have not solved problems, many of them related to the explosion in the number of frequent patterns found in the data set. This happens because frequent patterns conform to the anti-monotony property, which says that if a pattern is frequent, all its sub-patterns are also. This way the solution, by having redundant information from patterns of low significance, does not add to the result information useful enough to justify its importance.
\par
This work presents a new methodology for obtaining patterns of interest in a data set that explores the concept of orthogonality - defined as the measure of how the elements of a set does not contribute with redundant information to the solution of a problem - and its application in associative classification, as a way to increase the effectiveness of a classifier, reducing the redundancy and ambiguity of the rules.