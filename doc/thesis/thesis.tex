\documentclass[a4paper,11pt,oneside,openright,final,msc,project]{ufmgthesis}
\usepackage[brazil]{babel}
\usepackage[latin1]{inputenc}
\usepackage[T1]{fontenc}
\usepackage{type1ec}
\usepackage{graphicx}
\usepackage[
	a4paper,
	portuguese,
	bookmarks=true,
	bookmarksnumbered=true,
	linktocpage,
	colorlinks
]{hyperref}
\usepackage{natbib}

\begin{document}

\ufmgthesis{
	title={Ortogonalidade Aplicada a Regras de Associa��o},
	author={Leandro Souza Costa},
	university={Universidade Federal de Minas Gerais},
	course={Ci�ncia da Computa��o},
	address={Belo Horizonte},
	date={2008-04-16},
	logo={img/brasao.eps},
	advisor={Wagner Meira Jr.}
	member={Marcos Andr� Gon�alves}{-}{Universidade Federal de Minas Gerais},
	member={Sandra de Amo}{-}{Universidade Federal de Uberl�ndia},
	abstract={Resumo}{resumo.tex},
%	dedication={dedicatoria.tex},
	ack={agradecimentos.tex}
}

\chapter{Introdu��o}

\section{Minera��o de Dados}
\subsection{Padr�es Frequentes}
\subsection{Regras de Associa��o}
\section{Ortogonalidade}
\section{Organiza��o do Documento}

\chapter{Algoritmos de Classifica��o}

\section{Fundamentos Te�ricos e Defini��es}
\section{Estrat�gias eager e lazy}
\section{M�tricas de Regras de Associa��o}

Falar sobre similaridade, cobertura, lift, leverage, etc.

\section{Trabalhos Relacionados}

Falar sobre o lazy

\chapter{Ortogonalidade}

\section{M�tricas de Ortogonalidade}
\subsection{Similaridade entre Padr�es}
\subsection{Cobertura de Transa��es}
\subsection{Cobertura de Classes}
\section{Estrat�gias de Ortogonalidade}
\subsection{Ortogonalidade por conjunto}
\subsection{Ortogonalidade Par-a-par}
\section{Classifica��o e Ortogonalidade}
\subsection{Utiliza��o de Ortogonalidade no Lazy}
\subsection{Heur�stica de Obten��o de Conjuntos Ortogonais}
\section{Trabalhos Relacionados (ORIGAMI) (???)}
\subsection{Defini��o de alfa-ortogonalidade}
\subsection{Estrat�gia de Ortogonalidade (algoritmo)}
\subsection{Adapta��o do algoritmo}

\chapter{Experimentos e Resultados}

\section{O Aplicativo machado}
\section{Compara��o Ortogonal x N�o Ortogonal}
\section{Compara��o Ortogonal x Lazy}
\section{Compara��o Ortogonal x ORIGAMI}

\chapter{Resumo e Trabalhos Futuros}

\section{Resumo}
\section{Trabalhos Futuros}

\ufmgbibliography{thesis}

\end{document}
