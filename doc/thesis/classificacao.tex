\chapter{Algoritmos de Classifica��o}

O artigo \cite{DBLP:conf/icde/ChengYHH07} possui uma se��o dizendo por qu� padr�es frequentes s�o bons para classifica��o.

\section{Fundamentos Te�ricos e Defini��es}



\section{Estrat�gias eager e lazy}


\section{M�tricas de Regras de Associa��o}

Falar sobre similaridade, cobertura, lift, leverage, etc. \\
fonte: http://www.daylight.com/meetings/emug01/Bradshaw/Similarity/YAMS.html \\
fonte: http://wwwai.wu-wien.ac.at/\%7Ehahsler/research/association\_rules/measures.html

\section{Trabalhos Relacionados}

Falar de \cite{DBLP:conf/kdd/KnobbeH06}, que obt�m, de um conjunto de itens, um itemset que particione uma base de dados o mais uniformemente poss�vel.
Falar de \cite{DBLP:conf/icde/LentSW97} ??? artigo que realiza agrupamento de regras de associa��o num espa�o bi-dimensional.
Falar de \cite{DBLP:conf/kdd/XinCYH06}, que extrai top-k padr�es minimizando a redund�ncia de um conjunto de padr�es frequentes.
Falar sobre o ORIGAMI \cite{zaki07origami}.
Falar sobre o lazy \cite{Veloso06Lazy}
