Minera��o de padr�es freq�entes � um dos temas mais explorados da minera��o de dados, assumindo um papel essencial em muitas tarefas que possuem, como objetivo, encontrar padr�es de determinado interesse numa base. Entretanto, grande parte das solu��es encontradas ainda possuem problemas n�o solucionados, e, em parte, relacionados com a caracter�stica do conjunto de padr�es selecionados. A classifica��o associativa, em especial, possui problemas relacionados com a explos�o do n�mero de padr�es freq�entes, o que faz com que um grande n�mero de regras redundantes ou amb�guas seja gerado. Esta disserta��o apresenta uma nova metodologia de obten��o de padr�es de interesse numa base de dados que explora o conceito de ortogonalidade. Em seguida, apresenta um modelo de classifica��o associativa que gera regras a partir de um conjunto de padr�es freq�entes e ortogonais com a inten��o de aumentar a efic�cia do classificador, diminuindo a redund�ncia e a ambig�idade das regras.