\section{Conclus�o}

\begin{frame}{Conclus�es}
	\begin{itemize}[<+-| alert@+>]
		\item As abordagens baseadas em ortogonalidade obtiveram resultados semelhantes aos da abordagem cl�ssica:
		\begin{itemize}[<+-| alert@+>]
			\item Considerando os melhores par�metros para cada base, as m�dias de acur�cia obtidas para as abordagens LAC, OLAC e ORIGAMI foram, respectivamente, $0.843$, $0.840$ e $ 0.839$;
			\item Considerando os melhores par�metros para a m�dia dos resultados, as m�dias de acur�cia obtidas para as abordagens LAC, OLAC e ORIGAMI foram, respectivamente, $0.808$, $0.813$ e $0.782$.
		\end{itemize}
		\item A quantidade de padr�es utilizados na gera��o das regras nas abordagens ortogonais foi bem menor que na abordagem cl�ssica:
		\begin{itemize}[<+-| alert@+>]
			\item Considerando os melhores par�metros para cada base, as quantidades m�dias de padr�es utilizados pelas abordagens LAC, OLAC e ORIGAMI foram, respectivamente, $213$, $12$ e $12$;
			\item Considerando os melhores par�metros para a m�dia dos resultados, as quantidades de padr�es utilizados pelas abordagens LAC, OLAC e ORIGAMI foram, respectivamente, $19$, $12$ e $1$.
		\end{itemize}
		\item Conseq�entemente, a quantidade de regras geradas nas abordagens ortogonais tamb�m foi menor que na abordagem cl�ssica:
		\begin{itemize}[<+-| alert@+>]
			\item Considerando os melhores par�metros para cada base, as quantidades m�dias de regras geradas pelas abordagens LAC, OLAC e ORIGAMI foram, respectivamente, $628$, $25$ e $23$;
			\item Considerando os melhores par�metros para a m�dia dos resultados, as quantidades de regras geradas pelas abordagens LAC, OLAC e ORIGAMI foram, respectivamente, $51$, $31$ e $1$.
		\end{itemize}
	\end{itemize}
\end{frame}

% Trabalhos Futuros