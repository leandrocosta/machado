\section{Conclusions}
\subsection{Results}

\begin{frame}{Accuracy}
	\begin{itemize}[<+-| alert@+>]
		\item With orthogonality based approaches we got good results, very close to the classical approach one:
		\begin{itemize}[<+-| alert@+>]
			\item Considering the best parameters for each dataset, the values for accuracies obtained by LAC, OLAC and ORIGAMI were, respectively, $0.843$, $0.840$ e $ 0.839$;
			\item Considering the best parameters for the average of the results, the values for accuracy obtained by LAC, OLAC and ORIGAMI were, respectively, $0.808$, $0.813$ e $0.782$.
		\end{itemize}
	\end{itemize}
\end{frame}

\begin{frame}{Patterns}
	\begin{itemize}[<+-| alert@+>]
		\item The number of patterns used to generate the rules by the orthogonal approaches were lower than by the classical approach:
		\begin{itemize}[<+-| alert@+>]
			\item Considering the best parameters for each dataset, the number of patters used by LAC, OLAC and ORIGAMI were, respectively, $213$, $12$ e $12$;
			\item Considering the best parameters for the average of the results, the number of patterns used by LAC, OLAC and ORIGAMI were, respectively, $19$, $12$ e $1$.
		\end{itemize}
	\end{itemize}
\end{frame}

\begin{frame}{Rules}
	\begin{itemize}[<+-| alert@+>]
		\item The number of rules generated by the orthogonal approaches were much lower than by the classical approach:
		\begin{itemize}[<+-| alert@+>]
			\item Considering the best parameters for each dataset, the number of rules generated by LAC, OLAC and ORIGAMI were, respectively, $628$, $25$ e $23$;
			\item Considering the best parameters for the average of the results, the number of rules generated by LAC, OLAC and ORIGAMI were, respectively, $51$, $31$ e $1$.
		\end{itemize}
	\end{itemize}
\end{frame}

\begin{frame}{Other Results}
	\begin{itemize}[<+-| alert@+>]
		\item The orthogonality metric based in pattern structure obtained the best results;
		\item The association rule metrics that obtained the best results were conviction by LAC and OLAC and confidence by ORIGAMI;
		\item Most of fails found in classification based in orthogonality was not caused by the low orthogonality metric. They were found because sometimes, the different patterns used, induced the results to the wrong class.
	\end{itemize}
\end{frame}

\subsection{Future Works}

\begin{frame}{Next Steps}
	\begin{itemize}[<+-| alert@+>]
		\item Use orthogonality in another points of a classification algorithm;
		\item Research for new heuristics of orthogonal patterns extraction, in order to improve performance;
		\item Research for new algorithms of frequent pattern mining that consider  orthogonality while generating the frequent patterns;
		\item Use of a hybrid approach OLAC-ORIGAMI.
	\end{itemize}
\end{frame}

\subsection{End}
\begin{frame}[c]
\begin{center}
\Large
\alert {Questions?}
\Large
\end{center}
\end{frame}
