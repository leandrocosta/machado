\section{Exemplo de Execu��o da Heur�stica de Obten��o de Conjuntos Ortogonais}

\setbeamercovered{invisible}

\begin{frame}{Exemplo de Execu��o}

	\begin{overprint}
		\onslide<2-5>
			Padr�es Freq�entes: \{\alert<5>{ABC},BC,A,B\}
		\onslide<6-8>
			Padr�es Freq�entes: \{\st{ABC},\alert<8>{BC},A,B\}
		\onslide<9-14>
			Padr�es Freq�entes: \{\st{ABC},\st{BC},\alert<11-14>{A},B\}
		\onslide<15-22>
			Padr�es Freq�entes: \{\alert<22>{ABC},\st{BC},\st{A},\alert<16-19>{B}\}
		\onslide<23-28>
			Padr�es Freq�entes: \{\st{ABC},\st{BC},\st{A},\alert<25-28>{B}\}
		\onslide<29>
			Padr�es Freq�entes: \{\st{ABC},BC,\st{A},\st{B}\}
		\onslide<30>
			Padr�es Freq�entes: \{ABC,BC,A,B\}
	\end{overprint}
	
	\begin{overprint}
		\onslide<3-5>
			Padr�es Ortogonais: \{\}
		\onslide<6-8>
			Padr�es Ortogonais: \{ABC\}
		\onslide<9-14>
			Padr�es Ortogonais: \{\alert<12-14>{ABC},BC\}
		\onslide<15-22>
			Padr�es Ortogonais: \{A,\alert<17-19>{BC}\}
		\onslide<23-28>
			Padr�es Ortogonais: \{A,\alert<26-28>{BC},ABC\}
		\onslide<29>
			Padr�es Ortogonais: \{A,B,ABC\}
	\end{overprint}
	\begin{overprint}
		\onslide<4-9>
			Ortogonalidade: $-$
		\onslide<10-14>
			Ortogonalidade: \alert<14>{$0.33$}
		\onslide<15-23>
			Ortogonalidade: \alert<19-20>{$1$}
		\onslide<24-28>
			Ortogonalidade: \alert<28-29>{$0.5$}
		\onslide<29>
			Ortogonalidade: \alert<29>{$0.67$}
	\end{overprint}
	
	\hspace{1cm}
	
	\begin{overprint}
		\onslide<13-14>
			Padr�es Ortogonais (c): \{A,BC\}
		\onslide<18-19>
			Padr�es Ortogonais (c): \{A,B\}
		\onslide<27-28>
			Padr�es Ortogonais (c): \{A,B,ABC\}
	\end{overprint}
	\begin{overprint}
		\onslide<14>
			Ortogonalidade (c): \alert<14>{$1$}
		\onslide<19>
			Ortogonalidade (c): \alert<19>{$1$}
		\onslide<28>
			Ortogonalidade (c): \alert<28>{$0.67$}
	\end{overprint}
	
	\hspace{1cm}
	
	\begin{overprint}
		\onslide<7-20>
			Padr�es Ortogonais (r): \{ABC\}
		\onslide<21-30>
			Padr�es Ortogonais (r): \alert<30>{\{A,BC\}}
	\end{overprint}
	\begin{overprint}
		\onslide<7-20>
			Ortogonalidade (r): \alert<20>{$0$}
		\onslide<21-30>
			Ortogonalidade (r): \alert<29-30>{$1$}
	\end{overprint}

\end{frame}

\setbeamercovered{transparent}